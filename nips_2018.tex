\documentclass{article}

% if you need to pass options to natbib, use, e.g.:
% \PassOptionsToPackage{numbers, compress}{natbib}
% before loading nips_2018

% ready for submission
\usepackage{nips_2018}

% to compile a preprint version, e.g., for submission to arXiv, add
% add the [preprint] option:
% \usepackage[preprint]{nips_2018}

% to compile a camera-ready version, add the [final] option, e.g.:
% \usepackage[final]{nips_2018}

% to avoid loading the natbib package, add option nonatbib:
% \usepackage[nonatbib]{nips_2018}

\usepackage[utf8]{inputenc} % allow utf-8 input
\usepackage[T1]{fontenc}    % use 8-bit T1 fonts
\usepackage{url}            % simple URL typesetting
\usepackage{booktabs}       % professional-quality tables
\usepackage{amsfonts}       % blackboard math symbols
\usepackage{nicefrac}       % compact symbols for 1/2, etc.
\usepackage{microtype}      % microtypography

\usepackage{amsmath,amssymb,amsthm}
\usepackage[ruled,vlined,linesnumbered]{algorithm2e}
\usepackage{verbatim}
\usepackage{braket}
\usepackage{todonotes}
\usepackage{paralist}
\usepackage{caption}

\usepackage[pdfpagelabels,pagebackref,colorlinks,allcolors=blue]{hyperref}

\newcommand{\algnamecaps}{\textsc{BadGer RamPaGe}\xspace}
\newcommand{\algname}{\textsc{Badger Rampage}\xspace}
\newcommand{\algnameshort}{\textsc{BR}\xspace}


\title{Multi-Dimensional Balanced Graph Partitioning via Gradient Descent}

% The \author macro works with any number of authors. There are two
% commands used to separate the names and addresses of multiple
% authors: \And and \AND.
%
% Using \And between authors leaves it to LaTeX to determine where to
% break the lines. Using \AND forces a line break at that point. So,
% if LaTeX puts 3 of 4 authors names on the first line, and the last
% on the second line, try using \AND instead of \And before the third
% author name.

\newtheorem{theorem}{Theorem}[section]
\newtheorem{lemma}[theorem]{Lemma}
\newtheorem{proposition}[theorem]{Proposition}
\newtheorem{corollary}[theorem]{Corollary}
\newtheorem{claim}[theorem]{Claim}
\newtheorem{fact}[theorem]{Fact}
\newtheorem{conj}[theorem]{Conjecture}
\newtheorem{definition}{Definition}[section]
\newtheorem{open}{Open Problem}[section]
%\theoremstyle{definition}
\newtheorem{remark}[theorem]{Remark}

\newenvironment{proofof}[1]{\begin{proof}[of {#1}]}{\end{proof}}

\newcommand{\mypar}[1]{\smallskip \noindent {\bf {#1}.}}
\newcommand{\myparq}[1]{\smallskip \noindent {\bf {#1}}}

\DeclareMathOperator*{\argmax}{arg\,max}
\DeclareMathOperator*{\argmin}{arg\,min}
%\newenvironment{proof}[1][Proof]{\begin{trivlist}
%\item[\hskip \labelsep {\bfseries #1}]}{\end{trivlist}}
%\newenvironment{definition}[1][Definition]{\begin{trivlist}
%\item[\hskip \labelsep {\bfseries #1}]}{\end{trivlist}}
\newenvironment{example}[1][Example]{\begin{trivlist}
		\item[\hskip \labelsep {\bfseries #1}]}{\end{trivlist}}
% \newenvironment{remark}[1][Remark]{\begin{trivlist}
% \item[\hskip \labelsep {\bfseries #1}]}{\end{trivlist}}

\renewcommand{\qed}{\nobreak \ifvmode \relax \else
	\ifdim\lastskip<1.5em \hskip-\lastskip
	\hskip1.5em plus0em minus0.5em \fi \nobreak
	\vrule height0.75em width0.5em depth0.25em\fi}

\newcommand{\ie}{{\it i.e.,\ }}
\newcommand{\eg}{{\it e.g.,\ }}
\newcommand{\etal}{{\it et al.\,}}
\newcommand{\through}{,\ldots,}
\newcommand{\cala}{\mathcal A}
\newcommand{\calb}{\mathcal B}
\newcommand{\calgn}{{\mathcal G}_n}
%\newcommand{\HHH}{\mathcal H}
%\newcommand{\DDD}{\mathcal D}
%\newcommand{\XXX}{\mathcal X}
\newcommand{\cals}{\mathcal S}
%\newcommand{\RRR}{\mathcal R}
\newcommand{\eps}{\ensuremath{\varepsilon}}
\newcommand{\z}{\mathrm{z}}
\newcommand{\zo}{\{0,1\}}
\newcommand{\oo}{\{+1,-1\}}
\newcommand{\inv}{^{-1}}
\def\E{\mathop{\mathbb{E}}\displaylimits}
\def\poly{\mathop{\rm{poly}}\nolimits}
\def\Lap{\mathop{\rm{Lap}}\nolimits}
\newcommand{\Cauchy}{\operatorname{Cauchy}}
\newcommand{\inr}{\in_{\mbox{\tiny R}}}
\newcommand{\cP}{\mathcal P}

\newcommand{\vxi}{\mathbb{\xi}}
\newcommand{\vone}{\mathbf{1}}

\newcommand{\vx}{\mathbf x}
\newcommand{\vy}{\mathbf y}
\newcommand{\vp}{\mathbf p}
\newcommand{\ve}{\mathbf e}



\author{
  Dmitrii Avdiukhin\thanks{Use footnote for providing further
    information about author (webpage, alternative
    address)---\emph{not} for acknowledging funding agencies.} \\
  Department of Computer Science\\
  Indiana University\\
  Bloomington, IN, 47405 \\
  \texttt{davdyukh@iu.edu} \\
  \And
  Sergey Pupyrev\thanks{} \\
  Facebook \\
  Menlo Park, CA\\
  \texttt{spupyrev@gmail.com}
 \And
 Grigory Yaroslavtsev \thanks{}\\
 Department of Computer Science \\
 Indiana University\\
 Bloomington, IN, 47405 \\
 \texttt{grigory@grigory.us}
  %% Coauthor \\
  %% Affiliation \\
  %% Address \\
  %% \texttt{email} \\
  %% \AND
  %% Coauthor \\
  %% Affiliation \\
  %% Address \\
  %% \texttt{email} \\
  %% \And
  %% Coauthor \\
  %% Affiliation \\
  %% Address \\
  %% \texttt{email} \\
  %% \And
  %% Coauthor \\
  %% Affiliation \\
  %% Address \\
  %% \texttt{email} \\
}

\begin{document}
% \nipsfinalcopy is no longer used

\listoftodos

\tableofcontents

\newpage


\maketitle

\begin{abstract}
  The abstract paragraph should be indented \nicefrac{1}{2}~inch
  (3~picas) on both the left- and right-hand margins. Use 10~point
  type, with a vertical spacing (leading) of 11~points.  The word
  \textbf{Abstract} must be centered, bold, and in point size 12. Two
  line spaces precede the abstract. The abstract must be limited to
  one paragraph.
\end{abstract}

\section{Introduction}\label{sec:intro}

\todo{Shouldn't we talk about motivation?}
We give fast and scalable practical algorithms for the problem of partitioining large graphs into components of roughly the same size/weight according to multiple user-specified weight functions. This problem, referred to as \textit{multi-dimensional balanced graph partitioning} (see Section~\ref{sec:mdbgp-definitions} for formal definitions) arises in critical infrastructure applications which involve storage and processing of large graphs, including social networks. High-quality partitions help optimize load balancing in query processing, \todo{elaborate} etc.  
\todo{Talk about why multi-dimensional is particularly important vs. one-dimensional}
While a large body of work exists offering practical solutions for the one-dimensional version of the problem~\cite{KK95, DGRW12, UB13, TGRV14, ABM16, DKKOPS16, MLLS17, KKPPSAP17} (see also a recent survey~\cite{BMSSS16}), as well as on theoretical foundations of graph partitioning~\cite{KNS09,AFKRS14,MM14}\todo{cite a bunch of theory papers}, literature on principled and scalable approaches for the multi-dimensional case is quite sparse~\cite{}. In particular, if the weight functions are unrelated to each other, one can easily construct examples when no feasible solution exists that satisfies all balance constraints even for two weight functions. 
\todo{Say that in general no feasible solution might exist at all.}

\paragraph {Our contributions}
Let $G(V,E)$ be an $n$-vertex graph whose adjacency matrix is $A$. We introduce a family of algorithms for the multi-dimensional balanced graph partitioning problem by using the \textit{projected gradient descent method} on a standard relaxation which involves maximizing an $n$-dimensional non-convex quadratic function $f(\vx) = \vx^T A \vx$ subject to a constraint $\vx \in K$ for some convex body $K$ defined by the weight functions \footnote{While second-order methods could potentially give better performance in terms of partition quality, due to the large scale of our instances such methods are infeasible.}. See Section~\ref{sec:algorithm} for the exact description of the relaxation\todo{Maybe need a citation here to justify this is standard}.

While applying projected gradient descent to solve \todo{discuss how constrained is different from unconstrained} non-convex optimization problems subject to convex constraints is a well-studied approach in non-linear optimization~\cite{B99} (Section 2.3) and machine learning~\cite{JK17} (Section 6.6), one has to overcome several technical challenges to make it applicable to the multi-dimensional graph partitoning problem: 1) projection step is computationally expensive, 2) abundance of saddle points slows down convergence.

We show how to address the first challenge by designing ultra-efficient projection step algorithms tailored to the standard non-convex relaxation of the multi-dimensional balanced graph partitioning problem. Computationally the problem of finding the closest point in the 
 For balance according to one weight function our projection algorithm runs in time $O(n)$ where $n$ is the number of vertices in the graph. For two weight functions we show how to implement projection in time $O(n \log^2 n)$. For $k$ weight functions the time is \todo{TBD}.

In order to address the second challenge we use small perturbations to each intermediate point, where the perturbation vectors are sampled from a scaled $n$-dimensional Gaussian distribution. \todo{cite some literature on randomized PGD} We refer to this algorithm as \algname (Algorithm~\ref{alg:mdgp-rpgd}).
We show how the magnitude of Gaussian noise affects convergence properties of \algname by helping it escape from saddle points~\cite{} \todo{cite literature on escaping from saddle points using SGD, etc.}.

Our experimental results show that \algname can be scaled to graphs with billions of vertices and hundreds of billions of edges. \algname outperforms... \todo{write this}.

\paragraph{Previous work}

While one-dimensional balanced graph partitioning has been studied extensively and a number of tools exist, to the best of our knowledge none of the practical algorithms for this problem have been previously based on running gradient descent on a continuous relaxation\todo{We will have to triple check this by looking at various NIPS papers}.
Existing approaches are inherently discrete and are based on combinations of various discrete algorithms ideas with heuristics: combinatorial heuristics and greedy methods (e.g. METIS~\cite{KK95}, FENNEL~\cite{TGRV14}), branch-and-bound (e.g.~\cite{DGRW12}), label propagation and local search (e.g. balanced label propagration~\cite{UB13}, Social Hash Partitioner~\cite{KKPPSAP17}, Spinner~\cite{MLLS17}), as well as hybrid approaches (e.g. linear embedding method combined with various optimizations~\cite{ABM16}). See~\cite{BMSSS16} for a recent survey of this topic. Due to the combinatorial nature of these algorithms their generalizations to the multi-dimensional case appear to be non-straightforward without substantial losses in performance, while a continuous relaxation handles multiple balance constrains uniformly.

Vast literature exists on optimization of non-convex functions and the interest in this topic lately has been particularly high~\cite{}\todo{cite all sorts of classic and recent gradient descent papers}. However, in the constrained case when the optimization has to be performed over a convex body fairly little is known for the general case (see e.g. classic texts~\cite{B99,WN99,BV04}). We refer to the convex constrained non-convex optimization problem as CNOPT, as in Section 6.6 of a recent survey on non-convex optimization for machine learning~\cite{JK17}.

\begin{itemize}
%\item Discuss all papers on balanced graph partitioning again~\cite{KK95, DGRW12, UB13, TGRV14, ABM16,  DKKOPS16, MLLS17, KKPPSAP17}, survey~\cite{BMSSS16}.
\item Discuss all the standard non-convex optimization subject to convex constraints (CNOPT) literature again~\cite{B99} (Section 2.3),~\cite{JK17} (Section 6.6).
\item Discuss recent papers which use first-order methods for CNOPT and show how to escape saddle points in general and specific situations. Cite noisy SGD~\cite{GHJY15}, some polynomial time algortihm which converges to third-order local optimum~\cite{AG16}, matrix completion~\cite{GLM16}, second-order methods~\cite{SQW15}.
\item Discuss papers which use PGD for constrained non-convex optimization problems arising from graph partitioning~\cite{LRSST10}.\todo{Find more papers to cite here}.
\item Discuss why we can't use off the shelf QP solvers, like OSQP from Steven Boyd and others (they wouldn't scale to billions of vertices)~\cite{SBGBB17}.\todo{seems like they are only looking at the convex case, their number of non-zeros is at most $10^8$ it seems}
\end{itemize}
\todo{How about SDP approaches?}

\paragraph{Our techniques}



\section{Preliminaries}\label{sec:mdbgp-definitions}

We study multi-dimensional graph partitioning problems. The basic one-dimensional unweighted graph partitioning problem is as follows: 

\begin{definition}[\textsc{$(1 \pm \epsilon)$-Balanced $k$-Partition}]
	Given an input graph $G(V, E)$, an integer $k$ and a parameter $\epsilon > 0$ the goal is to find a partition of the vertex set $V$ into $k$ sets $V_1, \dots, V_k$ such that $|V_i| = \frac{(1\pm \epsilon) |V|}{k}$ for all $i \in [k]$. Among all such partitions the goal is to find one that maximizes the number of edges whose both endpoints are contained within some part of the partition. 
\end{definition}

The more general weighted $d$-dimensional version is defined by a collection of $d$ weight functions $w_1, \dots, w_d$ where each $w_i \colon V \to \mathbb R^+$ is a real-valued weight function. For a set $S \subseteq V$ we use notation $w_i(S) \equiv \sum_{v \in S} w_i(v)$. For example, the unweighted case above corresponds to $d = 1$ and $w_1(v) = 1$ for all $v \in V$.

\begin{definition}[\textsc{Multi-dimensional Weighted $(1 \pm \epsilon)$-Balanced $k$-Partition}]
	Given an input graph $G(V, E)$, an integer $k$ and a parameter $\epsilon > 0$ the goal is to find a partition of the vertex set $V$ into $k$ sets $V_1, \dots, V_k$ such that for each $j \in [d]$ it holds that  $w_j(V_i) = \frac{(1\pm \epsilon) w_j(V)}{k}$ for all $i \in [k]$. Among all such partitions the goal is to find one that maximizes the number of edges whose both endpoints are contained within some part of the partition. 
\end{definition}


\section{\algname: Randomized Projected Gradient Descent Algorithm}\label{sec:algorithm}

The standard integer quadratic program for the weighted balanced graph $2$-partitioning problem is:
\begin{align*}
&\text{Maximize:  }  && \frac12  \sum_{(i, j) \in E} (x_i x_j + 1) && \\
&\text{Subject to:} && \left|\sum_{j = 1}^n w_i(j) x_j\right| \le \epsilon \sum_{j = 1}^n w_i(j)  && \forall i \in [d]\\
&&& x_i \in \{-1,1\} && \forall i \in V
\end{align*}

Dropping the additive term in the objective and relaxing the integrality constraints we have a relaxation:
\begin{align*}
&\text{Maximize:}  && \vx^T A \vx && \\
&\text{Subject to:} && \left|\sum_{j = 1}^n w_i(j) x_j\right| \le \epsilon \sum_{j = 1}^n w_i(j)  && \forall i \in [d]\\
&&& x_i \in [-1,1] && \forall i \in V
\end{align*}

Denoting $f(\vx) = \vx^T A \vx$ we have the gradient $\nabla f(\vx) = A \vx$ and Hessian $H_f = A$.
%Denoting the objective as $f(\vx) = \sum_{(i,j)\in E} x_i x_j$ and taking derivatives we have:
%\begin{align*}
%\nabla f(\vx)_i &= \sum_{(i,j) \in E} x_j \\
%(H_f)_{i,j} &= A_{ij}
%\end{align*}

\todo{cite Recht et al. NIPS'11 paper suggested by Kostya }
We propose the following general algorithm for the multi-dimensional weighted balanced graph partitioning problem.
Let $\mathcal B_\infty = \{\vx \in \mathbb R^n | \forall i \colon \vx_i \in [-1,1]\}$.
For $i \in [d]$ let $\mathcal S^i_\epsilon = \{\vx \in \mathbb R^n  | |\sum_{j = 1}^n w_i(j) \vx_j | \le \epsilon \sum_{j = 1}^n w_i(j)\}$.

\begin{algorithm}[H]\label{alg:mdgp-rpgd}
	\SetKwInOut{Input}{input}
	\SetKwInOut{Output}{output}
	\SetKwRepeat{Do}{do}{while}
	\Input{Graph $G(V,E)$, integer $k$, real value $\epsilon \in [0,1]$, weight functions $w_1, \dots, w_d \colon V \to \mathbb R^+$}
	\Output{$(1\pm \epsilon)$-balanced partition w.r.t $w_1, \dots, w_d$ of $V$ into $(V_1, V_2)$.}
	\caption{\algname ($d$-Dimensional Balanced Graph $2$-Partitioning via Randomized Projected Gradient Descent)}\label{alg:main}
	$\vx_0 = 0, t = 0$ \\	
	\Do{$\|\vx_t - \vx_{t + 1}\|_2 > \theta$}{
		$\vx'_{t} = \vx_t + \eta_t N(0,1)$\\
		$\vy_{t + 1} = (I + \gamma_t A) \vx'_t $\\
		$\vx_{t + 1} = \argmin_{\vx \in K} \|\vy_{t + 1} - \vx\|_2,$ where $K = \mathcal B_\infty \bigcap_{j = 1}^d \mathcal S^j_\epsilon$\\
		$t = t + 1$ \\ 
	} 
\end{algorithm}
\section{Projection step}\label{sec:projection}
\todo{cite some paper which focus on efficincy of projection step in PGD}

\subsection{Approximate projection for $d = 1$}

Formally, we have the following optimization problem. Given a fixed vector $\vy \in \mathbb R^n$ and allowed imbalance $\epsilon$:
\begin{align*}
& \text{Minimize:} && f(\vx) = \|\vx - \vy\|_2^2&& \\ 
& \text{Subject to:} && g_i = x_i^2 - 1 \le 0 &&\\
&&& h_+ = \sum_{i = 1}^n x_i - \epsilon \le 0 && \\
&&& h_- = -\sum_{i = 1}^n x_i - \epsilon \le 0 && \\
\end{align*}

By KKT:
$$(\vy - \vx) = \sum_{i = 1}^n \mu_i x_i \ve_i + (\mu_+ - \mu_-) \sum_{i = 1}^n \ve_i.$$
I.e. for each coordinate we have $y_i - x_i = \mu_i x_i + \mu_+ - \mu_-$ where and $\mu_i, \mu_+, \mu_- \ge 0$.
Complementary slackness gives $\mu_i (x_i^2 -1) = 0$, i.e. for each $i$ either $|x_i| = 1$ or $\mu_i = 0$.
Moreover, we have additional slackness constraints:
\begin{itemize}
	\item $\mu_+ (\sum_{i = 1}^n x_i - \epsilon) = 0$.
	\item $\mu_- (-\sum_{i = 1}^n x_i - \epsilon) = 0$.
\end{itemize}
%Now we can solve this problem like exact projection, with $\lambda = \mu_+ - \mu_-$ and $\sum h_i \in [\sum y_i - \epsilon; \sum y_i + \epsilon]$.

Consider $3$ cases:
\begin{enumerate}
	\item $\sum x_i = \epsilon$. In this case $\mu_- = 0$, and the first slackness constraint equals $y_i - x_i = \mu_i x_i + \mu_+$. Now this problem equals to exact projection, described in the previous section, with $\lambda=\mu_+$ and $\sum h_i = \sum y_i - \epsilon$.
	\item $\sum x_i = -\epsilon$. This case is similar to the previous one with $\lambda=-\mu_-$ and $\sum h_i = \sum y_i + \epsilon$.
	\item $\mu_+ = \mu_- = 0$. The first slackness constraint equals $y_i - x_i = \mu_i x_i$. This case is just projection on the hypercube, with restriction $\sum h_i \in (\sum y_i - \epsilon; \sum y_i + \epsilon)$.
\end{enumerate}
In all cases we have $y_i - x_i = \mu_i x_i + \lambda$, and possible values of $\lambda$ are disjoint.
Since $\sum h_i$ is increasing, only one of this options can be satisfied.
For example, if case one is satisfied, then for $\lambda < 0$ $\sum h_i(\lambda) = \sum y_i + \epsilon$.
Since $\sum h_i(\lambda)$ is increasing, for $\sum h_i(0) \ge \sum h_i(\lambda) = \sum y_i + \epsilon$, and therefore $\sum h_i(0) \notin (\sum y_i - \epsilon; \sum y_i + \epsilon)$.

Therefore, to find lambda we can use the following algorithm:
\begin{itemize}
	\item Try the case $\lambda = 0$. If $\sum h_i(0) \in (\sum y_i - \epsilon; \sum y_i + \epsilon)$, then return the corresponding $\vx$. Otherwise select the part for binary search:
	\begin{itemize}
		\item If $h(0) \ge \sum y_i + \epsilon$, search $\lambda$ in $(-\infty; 0]$ with $\sum h_i(\lambda) = \sum y_i + \epsilon$. 
		\item If $h(0) \le \sum y_i - \epsilon$, search $\lambda$ in $[0; +\infty)$ with $\sum h_i(\lambda) = \sum y_i - \epsilon$. 
	\end{itemize}
\end{itemize}


\subsection{Exact projection for $d = 2$}
\todo{change this to support approximate projection}
If we want to project on the intersection of two hyperplanes: $\sum_i w_i x_i = 0$ and $\sum_i w'_i x_i = 0$ then we can do this as follows. 
Using parameters $\lambda$ and $\lambda'$ as above we can still use the following algorithm setting $\gamma_i = \lambda w_i + \lambda' w'_i$:
\begin{enumerate}
	\item $(y_i \ge 1 + \gamma_i)$. Set $x_i = 1$.
	\item $(y_i \in (-1 + \gamma_i, 1 + \gamma_i))$. Set $x_i = y_i - \gamma_i$.
	\item $(y_i \le -1 + \gamma_i)$. Set $x_i = -1$.
\end{enumerate}

Now we have two balance functions: $h = \sum w_i x_i$ and $h' = \sum w'_i x_i$.
The change in $h$ after projection is expressed as:
\begin{align*}
\sum_i w_i y_i - \sum_i w_i x_i = &\sum_{i \colon y_i \ge 1 + \gamma_i} w_i (y_i - 1)
+ \sum_{i \colon y_i \in (-1+\gamma_i, 1+\gamma_i)} w_i \gamma_i
+ \sum_{i \colon y_i \le 1 - \gamma_i} w_i (1 + y_i) = \sum_{i} h_i(\lambda, \lambda'),
\end{align*}
where each $h_i$ is the following function:
\begin{align*}
h_i(\lambda, \lambda') = \begin{cases}
w_i (y_i - 1)  & \text{ if } \lambda w_i + \lambda' w'_i < y_i - 1 \\ 
w_i (\lambda w_i + \lambda' w'_i) & \text{ if } \lambda w_i + \lambda' w'_i \in [y_i - 1, y_i + 1]\\
w_i (y_i + 1) & \text{ if } \lambda w_i + \lambda' w'_i > y_i + 1 \\
\end{cases}
\end{align*}
Analogously, the difference between $h'$ can be expressed as $\sum h'_i$, where
\begin{align*}
h'_i(\lambda, \lambda') = \begin{cases}
w_i' (y_i - 1)  & \text{ if } \lambda w_i + \lambda' w_i' < y_i - 1 \\ 
w_i' (\lambda w_i + \lambda' w_i')  & \text{ if } \lambda w_i + \lambda' w_i' \in [y_i - 1, y_i + 1]\\
w_i' (y_i + 1) & \text{ if } \lambda w_i + \lambda' w_i' > y_i + 1 \\
\end{cases}
\end{align*}
Denote $h(\lambda, \lambda') = \sum h_i(\lambda, \lambda')$ and $h'(\lambda, \lambda') = \sum h'_i(\lambda, \lambda')$. We want to find $\lambda$ and $\lambda'$ such that $h(\lambda, \lambda') = \sum w_i y_i$ and $h'(\lambda, \lambda') = \sum w'_i y_i$.
We will show that we can use binary search for $\lambda$ with binary search for $\lambda'$ inside.
\begin{theorem}
	Consider the situation when $\lambda$ is fixed. Denote the \emph{maximum} $\lambda'$ such that $h(\lambda, \lambda') = \sum w_i y_i$ as $root(\lambda)$. The same way denote the \emph{maximum} $\lambda'$ such that $h'(\lambda, \lambda') = \sum w'_i y_i$ as $root'(\lambda)$. Then
	\[\lim_{\lambda \to +\infty}(root(\lambda) - root'(\lambda))
	\cdot \lim_{\lambda \to -\infty}(root(\lambda) - root'(\lambda))
	\le 0\]
\end{theorem}
Our goal is to find $\lambda$ such that $root(\lambda) - root'(\lambda) = 0$, meaning that there exist $\lambda'$, such that $h(\lambda, \lambda') = \sum w_i y_i$ and $h'(\lambda, \lambda') = \sum w'_i y_i$.
Since function $dif (\lambda) = root(\lambda) - root'(\lambda)$ is continuous (since it's piecewise-linear and continuous near borders??) and we can find points with different signs, we can find its root using binary search. (TODO: I think I know two pointers solution for this case). 

Note that there can be several $\lambda'$, corresponding to one $\lambda$. By selecting maximum value $root(\lambda)$ becomes unique. Now we will prove the theorem.
\begin{proof}
	For the proof we will use a geometric approach.
	We consider a two-dimensional plane $(\lambda, \lambda')$ and the following regions: $\lambda w_i + \lambda' w_i'$ for all $i$. We will show that when $\lambda \to +\infty$, $root(\lambda)$ form a line, lying in some region and parallel to its borders.
	
	First, note that there are only finite number of intersections between regions. Non-empty intersections can be of two following types:
	\begin{enumerate}
		\item Unbounded intersections. Each region corresponds to an area between two parallel lines, two regions are \emph{parallel} if their border lines are parallel.
		Then non-empty intersection of the regions is unbounded if all the regions are parallel.
		\item Bounded intersection. If some of regions are not parallel, the intersection is bounded.
	\end{enumerate}
	We consider the case when there are no unbounded intersections of more than one region.
	(TODO: consider another case.)
	By monotonicity and piecewise-linearity of $h$ and $h'$ $root$ and $root'$ are also piecewise-linear functions. Since region borders are lines, for large enough $\lambda$ $root(\lambda)$ entirely lies in one region or between them (TODO: show this).
	
	Consider function $h$. Sort all regions by its angle $k_i = - \frac {w_i'} {w_i}$.
	Change numeration of coordinates $\set{h_i}$ in such way that $k_i$ are decreasing.
	Then the following is true. When point $(\lambda, \lambda')$ belongs to $i$-th region, for large enough $\lambda$:
	\begin{itemize}
		\item For all $h_j$ where $j < i$ the third case is satisfied (since $(\lambda, \lambda')$ is above this region), namely $\lambda w_i + \lambda' w'_i > y_i + 1$ and $h_i(\lambda, \lambda') = w_i (y_i + 1)$.
		\item For all $h_j$ where $j > i$ the first case is satisfied (since $(\lambda, \lambda')$ is below this region), namely $\lambda w_i + \lambda' w'_i < y_i - 1$ and $h_i(\lambda, \lambda') = w_i (y_i - 1)$.
		\item For $i$-th region itself we know that $\lambda w_i + \lambda' w'_i \in [y_i - 1; y_i + 1]$. Therefore, $\lambda w_i + \lambda' w'_i = y_i + c$, where $c \in [-1; 1]$.
		$h_i(\lambda, \lambda') = w_i (y_i + c)$
	\end{itemize}
	Computing $h(\lambda, \lambda')$ gives us
	\[h(\lambda, \lambda') = \sum\limits_{j < i} w_i (y_i + 1) + w_i (y_i + c) + \sum\limits_{j > i} w_i (y_i - 1) =\]
	\[= \sum\limits_i w_i y_i + \sum\limits_{j < i} w_i + w_i c - \sum\limits_{j > i} w_i\]
	In case when $(\lambda, \lambda')$ lies between $(i-1)$-th and $i$-th regions, there is no $w_i c$ term and 
	\[h(\lambda, \lambda') = \sum\limits_i w_i y_i + \sum\limits_{j < i} w_i - \sum\limits_{j \ge i} w_i\]
	So, the value of $h$ doesn't change between two regions, which also follows from $h_i$ definition (it is constant on each side of $i$-th region). Also note that it matches the value of $h$ on the borders of $(i-1)$-th and $i$-th region, because values of $c$ are $1$ and $-1$ respectively.
	
	Now we can find $root(\lambda)$ when $\lambda \to \infty$. We need to find $i$ and $c$ such that
	\[\sum\limits_i w_i y_i = h(\lambda, \lambda') = \sum\limits_i w_i y_i + \sum\limits_{j < i} w_i + w_i c - \sum\limits_{j > i} w_i \iff \sum\limits_{j < i} w_i + w_i c - \sum\limits_{j > i} w_i = 0\]
	Note that since all $w_i$ are non-negative, there is an only way to split such sum, but there can be two representations when $c \in \set{-1; 1}$ (i.e. when $(\lambda, \lambda')$ is between regions).
	In such case the maximum value of $\lambda'$ is achieved when $c = -1$ (note that the maximum value is achieved on the left border of the right region, which has greater index).
	Denote such $i$ and $c$ as $i_+$ and $c_+$ respectively.
	(TODO: handle the case when $(\lambda, \lambda')$ is not between regions, but on the left or on the right of all of them.)
	(TODO: handle the case when some regions are horizontal or vertical.)
	(TODO: draw some pictures.)
	
	Consider $root(\lambda)$ when $\lambda \to -\infty$. By the similar reasoning we achieve the following equation for $i$ and $c$:
	\[-\sum\limits_{j < i} w_i + w_i c + \sum\limits_{j > i} w_i = 0
	\iff \sum\limits_{j < i} w_i - w_i c - \sum\limits_{j > i} w_i = 0 \]
	As can be seen, $i=i_+$ and $c=-c_+$ satisfy this equation.
	If $c_+ \in (-1; 1)$ then pair $(i_-, c_-) = (i_+, -c_+)$ is a solution, corresponding to the unique $\lambda'$.
	Otherwise $c_+=-1 \iff -c_+=1$ and we should assign $i_- = i-1$ and $c_-=-1$ (note that the maximum value is achieved on the left border of the right region, which has smaller index).
	
	By the same reasoning for $h'$ we have to solve the following equations
	\[
	\begin{aligned}
	\sum\limits_{j < i} w_i + w_i c - \sum\limits_{j > i} w_i = 0 \\
	-\sum\limits_{j < i} w_i + w_i c + \sum\limits_{j > i} w_i = 0
	\end{aligned}
	\]
	Denote the solution of the first equation as $(i'_+, c'_+)$.
	Then the solution of the second equation is
	\[
	(i'_-, c'_-)
	\begin{cases}
	(i'_+, -c'_+)  & \text{ if } c'_+ \in (-1; 1) \\
	(i'_+-1, -1)   & \text{ if } c'_+ = -1
	\end{cases}
	\]
	If $i_+ = i'_+$ and $c_+ = c'_+)$ then the theorem is proved, since $\lim_{\lambda \to +\infty}(root(\lambda) - root'(\lambda)) = 0$.
	Otherwise, w.l.o.g. assume that $i_+ < i'_+$ or ($i_+ = i'_+$ and $c_+ < c'_+)$).
	In this case $\lim_{\lambda \to +\infty}(root(\lambda) - root'(\lambda)) < 0$, since less pair $(i,c)$ corresponds to less $\lambda'$ when $\lambda \to \infty$.
	
	Our goal is to show that $\lim_{\lambda \to -\infty}(root(\lambda) - root'(\lambda)) \ge 0$, and namely that $i_- < i'_-$ or ($i_- = i'_-$ and $c_- \ge c'_-$).
	We need to consider the following cases:
	\begin{itemize}
		\item $i_+ < i'_+$ and $c'_+ = -1$. Then $i'_- = i'_+-1 \ge i_-$ and $c'_- = -1 \le c_-$.
		\item $i_+ < i'_+$ and $c'_+ \in (-1; 1)$. Then $i'_- = i'_+ > i_-$.
		\item $i_+ = i'_+$ and $c'_+ = -1$. Impossible, since $c_+$ must be less than $c'_+$.
		\item $i_+ = i'_+$, $c_+ = -1$ and $c'_+ \in (-1; 1)$. Then $i'_- = i'_+ > i_+-1 = i_-$.
		\item $i_+ = i'_+$, $c_+, c'_+ \in (-1; 1)$. Then $i'_- = i_-$ and $c'_- < c_-$.
	\end{itemize}
\end{proof}
\begin{comment}
The intuition is that we might be able to use double binary search (outer search on $\lambda'$, inner search on $\lambda$). In the outer search we fix $\lambda'$ and do binary search on $\lambda$ for both functions $h$ and $h'$. We would like both $h$ and $h'$ to take zero value at the same point $\lambda_*$ when $\lambda'$ is fixed.
If this doesn't happen then we will find two values $\lambda_*$ and $\lambda'_*$ for $h$ and $h'$ respectively. Depending on whether $\lambda_* > \lambda'_*$ or $\lambda_* < \lambda'_*$ we can reduce the search space in our outer binary search for $\lambda'$.
The tricky part is to show that this reduction is indeed correct.
In order to do this it suffices to show that when $\lambda' \to \pm \infty$ we have $\lambda_* > \lambda'_*$ and $\lambda_* < \lambda'_*$ respectively. Then by continuity our outer binary search will find a point where $\lambda_* = \lambda'_*$.

Let's assume for a moment that $n = 1$ so we just need to balance two functions $h(\lambda, \lambda')$ and $h'(\lambda, \lambda')$.
We have:
\begin{align*}
h(\lambda, \lambda') = \begin{cases}
w (1 - y)  & \text{ if } \lambda w + \lambda' w' < y - 1 \\ 
- (\lambda w + \lambda' w') w & \text{ if } \lambda w + \lambda' w' \in [y - 1, y + 1]\\
- w (1 + y) & \text{ if } \lambda w + \lambda' w' > y + 1 \\
\end{cases}\\
h'(\lambda, \lambda') = \begin{cases}
w' (1 - y)  & \text{ if } \lambda w + \lambda' w' < y - 1 \\ 
- (\lambda w + \lambda' w') w' & \text{ if } \lambda w + \lambda' w' \in [y - 1, y + 1]\\
- w' (1 + y) & \text{ if } \lambda w + \lambda' w' > y + 1 \\
\end{cases}
\end{align*}
\end{comment}
\section{Towards convergence analysis}\label{sec:convergence}
%If the domain of $f$ is restricted to $[-\Delta, \Delta]^{|V|}$ then the Lipschitz constant of $f \colon [-\Delta, \Delta]^{|V|} \to \mathbb R$ is:
%$$K = \sup_{\vx \in [-\Delta,\Delta]^{|V|}} \|\nabla f(\vx)\|_2 \le \Delta d \sqrt{|V|},$$
%where $d$ is the maximum degree in $G$.




\begin{lemma}[Bertsekas, Section 2.3.2]\label{lem:pgd-step}
	If $\|\nabla f(\vx) - \nabla f(\vy)\| \le L \|\vx - \vy\|$ for all $\vx, \vy \in \mathbb R^n$ and $0 < \gamma < 2/L$ then 
	\begin{align*}
	f(\vx_{t + 1}) - f(\vx_t) \ge \left(\frac{1}{\gamma} - \frac{L}{2}\right)\|\vx_t - \vx_{t+1}\|_2^2
	\end{align*}
	
\end{lemma}

\begin{proof}
	
	We use the following Descent Lemma:
	
	\begin{proposition}[Descent Lemma]
		Let $f \colon \mathbb R^n \to \mathbb R$ be continuously differentiable, and let $\vx$ and $\vy$ be two vectors in $\mathbb R^n$. If for all $t \in [0,1]$ it holds that $\|\nabla f(\vx + t\vy) - \nabla f(\vx)\| \le L t \|\vy\|$ where $L$ is a constant then:
		$$f(\vx + \vy) \ge f(\vx) + \vy^T \nabla f(\vx) - \frac{L}{2} \|\vy\|_2^2$$
	\end{proposition}
	\begin{proof}
		Let $t$ be a scalar and let $g(t) = f(\vx + t \vy)$. By the chain rule $(d g / dt)(t) = \vy^T \nabla f(\vx + t\vy)$.
		Then:
		\begin{align*}
		f(\vx + \vy) - f(\vx) &= g(1) - g(0) \\
		& = \int_0^1 \frac{d g}{d t}(g) dt \\
		& = \int_0^1 \vy^T \nabla f(\vx + t \vy) dt \\
		&\ge \int_0^1 \vy^T \nabla f(\vx) dt - \left|\int_0^1 \vy^T (\nabla f(\vx + t \vy) - \nabla f(\vx)) dt\right| \\
		&\ge \int_0^1 \vy^T \nabla f(\vx) dt  - \int_0^1 \|\vy\|_2 \|\nabla f(\vx + t \vy) - \nabla f(\vx)\|_2dt\\
		& \ge \vy^T \nabla f(\vx) - \|\vy\|_2 \int_0^1 L t \|\vy\|_2 dt \\
		& = \vy^T \nabla f(\vx) - \frac{L}{2} \|\vy\|_2^2
		\end{align*}
	\end{proof}
	
	
	By the Descent Lemma we have:
	\begin{align*}
	f(\vx_{t + 1}) - f(\vx_t) \ge \nabla f(\vx_t)(\vx_{t + 1} - \vx_t) - \frac{L}{2}\|\vx_{t + 1} - \vx_t\|_2^2.
	\end{align*}
	
	Note that since $K = \mathcal S_0 \cap \mathcal B_\infty$ is a convex body and $\vx_{k + 1}$ is a projection of $\vy_{t + 1}$ on $K$ for every $\vx \in K$ it holds that:
	$$(\vy_{t + 1} - \vx_{t + 1})(\vx - \vx_{t + 1}) \le 0.$$
	Applying this to $\vx = \vx_t$ and using the fact that $\vy_{t + 1} = \vx_t + \gamma \nabla f(\vx_t)$ we have:
	$$(\vx_t + \gamma \nabla f(\vx_t) - \vx_{t + 1})(\vx_t - \vx_{t + 1}) \le 0,$$
	which implies that $\nabla f(\vx_t) (\vx_{t+1} - \vx_{t}) \ge \frac{1}{\gamma}\|\vx_t - \vx_{t+1}\|_2^2$. Hence we have:
	\begin{align*}
	f(\vx_{k + 1}) - f(\vx_k) \ge \left(\frac{1}{\gamma} - \frac{L}{2}\right)\|\vx_t - \vx_{t+1}\|_2^2
	\end{align*}
	
	%\textbf{The part below I don't quite understand, I think something is missing.}
	%Thus if $\gamma < 2/L$, the RHS of the above is nonnegative so if $\{\vx_t\}$ has a limit point the LHS tends to $0$.
	%Therefore, $\|\vx_{t + 1} - \vx_t\| \to 0$, which implies that for every limit point $\bar \vx$ of $\{\vx_t\}$ we have $[\bar \vx + \gamma \nabla f(\bar \vx)]^+ = \bar \vx$, so $\bar \vx$ is stationary (here we use notation $[\vx]^+$ to denote the projection of a vector $\vx$ on $K$).
\end{proof}

\subsection{First step}


Let's analyze the first step. We show the following theorem:
\begin{theorem}
	Let $\lambda_{max} = \max(|\lambda_1|, |\lambda_n|)$ then
	if $\gamma = 1/\lambda_{max}$ then $\mathbb E[f(\vx_1)] \ge \frac{\eta^2 |E|}{2 \lambda_{max}}$ (assuming we don't have to round the coordinates\todo{Fix this!}).
\end{theorem}


Consider three points $\vx_0, \vx_1$ and $\vy_1$. These three points lie in a hyperplane $H'$ which is orthogonal to the hyperplane $H = \{\vx \colon \langle \vone, \vx \rangle = 0\}$, where $\vone = (\frac{1}{\sqrt{n}}, \dots, \frac{1}{\sqrt{n}})$.
Let $\vx_0'$ be the projection of $\vx_0$ on $H$.
Then $\vx_0'$ also lies in $H'$.

We have $\|\vx_1 - \vx_0\|_2^2 = \|\vx_0 - \vx'_0\|_2^2 + \|\vx_0' - \vx_1\|_2^2$.
Let $\vy_1'$ be the projection of $\vx_0$ on the line through $\vy_1$ and $\vx_1$.
Then $\|\vx_0 - \vy_1\|_2^2 = \|\vx_0 - \vy_1'\|_2^2 + \|\vy_1' - \vy_1\|_2^2$.
Since $\|\vx_0 - \vy_1'\|_2^2 = \|\vx_0' - \vx_1\|_2^2$ we have:
\begin{align*}
\|\vx_1 - \vx_0\|_2^2 = \|\vx_0 - \vx_0'\|_2^2 + \|\vx_0 - \vy_1\|_2^2 - \|\vy_1' - \vy_1\|_2^2.
\end{align*}

We now take expectations and make use of the Lemma~\ref{lem:first-pgd-step-geometry} which is proved below:
\begin{align*}
\mathbb E\left[\|\vx_1 - \vx_0\|_2^2\right] &= \eta^2 \left(1 + \gamma^2 \|A\|_F^2 - \gamma^2 \sum_{i = 1}^n \lambda_i^2 \langle \vone, v_i\rangle^2\right) \\
& \ge \eta^2 (1 + 2 \gamma^2 |E| - \gamma^2 \max(\lambda_1^2, \lambda_n^2))
\end{align*}

%The following is a standard fact about eigenvalues of the adjacency matrix (see e.g. Lovasz\todo{add citation}):
%\begin{proposition}
%If $\bar d = \frac{1}{n} \sum_{i = 1}^n d_i$ is the average degree and $\lambda_1$ is the largest eigenvalue of the adjacency matrix then:
%$$\max(\bar d,\sqrt{d_{max}})\le \lambda_1 \le d_{max}$$
%\end{proposition}
%
%Hence:
%\begin{align*}
%\mathbb E\left[\|\vx_1 - \vx_0\|_2^2\right] \ge \eta^2 (1 + \gamma^2 (|E| - d_{max}^2) )
%\end{align*}

By Lemma~\ref{lem:pgd-step} using the fact that for our function $f(\vx) = \vx^T A \vx$ we have $L \le \max(|\lambda_1|, |\lambda_n|)$ we obtain:
$$f(\vx_1) - f(\vx_0) \ge \left(\frac{1}{\gamma} - \frac{\max( |\lambda_1|, |\lambda_n|)}{2}\right) \|\vx_1 - \vx_0\|_2^2.$$

Setting $\gamma = 1/ \max(|\lambda_1|, |\lambda_n|)$ and taking expectations we have:
$$\mathbb E[f(\vx_1) - f(\vx_0)] \ge \frac{\eta^2|E|}{2 \max(|\lambda_1|, |\lambda_n|)}  $$

Finally note that:
$$\mathbb E[f(\vx_0)] = \mathbb E[\vx_0 A \vx_0] = \eta^2 \sum_{i = 1}^n \lambda_i = \eta^2 tr(A) = 0,$$
and hence the proof of the theorem follows.

It remains to prove Lemma~\ref{lem:first-pgd-step-geometry}.

\begin{lemma}\label{lem:first-pgd-step-geometry}
	If $A = \sum_{i = 1}^n \lambda_i v_i v_i^T$ is the eigendecomposition of $A$ where $v_i$'s form an orthonormal basis then:
	\begin{align*}
	&\mathbb E[\|\vx_0 - \vx_0'\|_2^2] = \eta^2 \\
	&\mathbb E[\|\vx_0 - \vy_1\|_2^2 ] = \eta^2 \gamma^2 \|A\|_F^2\\
	&\mathbb E[\|\vy_1' - \vy_1\|_2^2] = \eta^2 \gamma^2 \sum_{i = 1}^n \lambda_i^2 \langle \vone, v_i \rangle^2
	\end{align*}
\end{lemma}
\begin{proof}
	We have $\mathbb E[\|\vx_0 - \vx_0'\|_2^2] = \mathbb E[\langle \vone, \vx_0 \rangle^2]ß = \eta^2$, where the second equality follows by rotational symmetry of the Gaussian distribution. 
	
	%$$\mathbb E[(\sum_{i = 1}^n \frac{1}{\sqrt n} (\vx_0)_i)^2] = \frac{1}{n} \mathbb E[\sum_{i = 1}^n (\vx_0)_i^2] = \eta^2 $$
	
	We have: 
	\begin{align*}
	\mathbb E[\|\vx_0 - \vy_1\|_2^2] &= \mathbb E[\|\gamma A \vx_0\|_2^2] \\
	&= \mathbb E\left[\gamma^2 \vx_0^T A^2 \vx_0 \right]\\
	&= \mathbb E\left[\gamma^2 \vx_0^T \left(\sum_{i = 1}^n \lambda^2_i v_i v_i^T\right) \vx_0 \right]\\
	&= \gamma^2 \sum_{i = 1}^n \lambda^2_i \mathbb E\left[\langle v_i, \vx_0\rangle^2\right] \\
	&= \gamma^2 \eta^2 \|A\|_F^2,
	\end{align*}
	where in the last equality we use the fact that since each $v_i$ is a unit vector $\mathbb E[\langle v_i, \vx_0\rangle ] = \eta^2$ by the rotaional symmetry of the Gaussian distribution.
	
	Finally, we have:
	\begin{align*}
	\mathbb E[\|\vy_1' - \vy_1\|_2^2] &= \mathbb E[\langle \vone, \vy_1 - \vx_0 \rangle^2 ] \\
	&= \mathbb E[\langle \vone, \gamma A \vx_0 \rangle^2] \\
	&= \gamma^2 \mathbb E\left[\left(\vone^T \left(\sum_{i = 1}^n \lambda_i v_i v_i^T\right) \vx_0\right)^2\right] \\
	&= \gamma^2 \mathbb E\left[ \left(\sum_{i = 1}^n \lambda_i \langle\vone, v_i\rangle \langle v_i, \vx_0\rangle \right)^2 \right] \\
	&= \gamma^2 \left(\sum_{i = 1}^n \sum_{j = 1}^n\lambda_i \lambda_j \langle\vone, v_i\rangle \langle \vone, v_j\rangle \mathbb E\left[\langle v_i, \vx_0\rangle \langle v_j, \vx_0\rangle\right]\right)
	\end{align*}
	Note that since $v_i$ and $v_j$ are orthogonal for $i \neq j$ we have $\mathbb E\left[\langle v_i, \vx_0\rangle \langle v_j, \vx_0\rangle\right] = 0$. For $i = j$ we have $\mathbb E[\langle v_i, \vx_0\rangle^2] = \eta^2$ as before. Hence we have:
	\begin{align*}
	\mathbb E[\|\vy_1' - \vy_1\|_2^2] = \eta^2 \gamma^2 \sum_{i = 1}^n \lambda_i^2 \langle \vone, v_i \rangle^2
	\end{align*}
	
	
\end{proof}


\subsection{$t$-th step}

We will assume that noise is added in every step, i.e. the algorithm at every step looks as follows:
\begin{enumerate}
	\item Given input $\vx_t$ from the pervious step let $\vx_t' = \vx_t + \eta N(0,1)$.
	\item Let $\vy_{t + 1} = \vx_t' + \gamma A \vx_t'$
	\item Set $\vx_{t + 1} = \arg min_{\vx \in \mathcal S_0 \cap \mathcal B_\infty} \|\vy_{t + 1} - \vx\|$.
\end{enumerate}

We need an analog of Lemma~\ref{lem:first-pgd-step-geometry}.
\begin{lemma}
	\begin{align*}
	\mathbb E[\|\vy_{t + 1} - \vx_{t}'\|_2^2] \ge \gamma^2 \eta^2 \|A\|_F^2.
	\end{align*}
\end{lemma}
\newcommand{\vz}{\mathbf z}
\begin{proof}
	Let $\vz \sim N(0,1)$
	\begin{align*}
	\mathbb E[\|\vy_{t + 1} - \vx_{t}'\|_2^2] &= \mathbb E[\|\gamma A \vx_t'\|_2^2] \\ 
	&= \gamma^2 \mathbb E[\vx_t'^Tß A^2 \vx_t'] \\
	&= \gamma^2 \mathbb E[(\vx_t + \eta \vz)^T A^2 (\vx_t + \eta \vz)] \\
	& = \gamma^2 \left(\vx_t^T A^2 \vx_tß + 2 \eta \mathbb E[\vz^T A^2 \vx_t] + \eta^2 \mathbb E[\vz^T A^2 \vz]\right)
	\end{align*}
	We have $\mathbb E[\mathbb \vz^T A^2 \vz] = \|A\|_F^2$ as in the proof of Lemma~\ref{lem:first-pgd-step-geometry}.
	Furthermore, $\mathbb E[\vz^T A^2 \vx_t] = \mathbb E[\langle \vz, A^2 \vx_t\rangle] = 0$ where the second equality follows by the linearity of expectation using the fact that $\mathbb E[\vz_i] = 0$ for each $i$.
	
	We have $\vx_t^T A^2 \vx_t = \sum_{i = 1}^n \lambda_i^2 \langle v_i, \vx_t\rangle^2 \ge 0$ which completes the proof.
	
	
\end{proof}
\section{Experiments}\label{sec:experiments}
We design our experiments to understand how well our agorithm behaves on real-world datasets and how
it compares to the state-of-the-art approaches. As pointed out in Section~\ref{bla}, we are not
aware of a scalable approach for solving the {\it multidimensional} balanced partitioning. Hence, 
we present a comparison of \algnameshort with related techniques for {\it one-dimensional} variant
of the problem. For the multi-dimensional variant, discussed in Section ~\ref{bla}, we present...

For our experiments, we use three publicly available social networks 
and several large subgraphs of the Facebook friendship graph. We utilize the public graphs for which the results of the state-of-the-art minimum-cut paritioning are known. The private datasets serve to demonstrate scalability of our approach and its performance on real-world data. Our dataset is as follows.

\begin{compactitem}
    \item \texttt{LiveJournal} is an undirected version of the public social graph (snapshot from 2006) containing $4.8$ million vertices and $42.9$ million edges~\cite{UB13}.

    \item \texttt{Twitter} is a public graph of tweets, with about $41$ million vertices (twitter accounts) and $1.2$ billion edges (denoting followership)~\cite{KCHM10}.

    \item \texttt{Friendster} is another public social network whose minimum-cut partitioning is available~\cite{LinearEmbed}; it contains $65$ million edges and $1.8$ billion edges.

  \item \texttt{FB-X} are subgraphs of the Facebook friendship graph, where $X$ indicates the (approximate) number of edges; the data was anonymized before processing.
\end{compactitem}

\subsection{One-dimensional partitioning}
We evaluate our algorithm, denoted by \algnameshort and described in Section~\ref{}, with existing scalable approaches for graph partitioning. Recall that our primary goal is to design and implement a scalable 
algorithm that can run for very large graphs in ditributed setting.
The most relevant works are the label propagation-based approaches by Ugander and Backstrom~\cite{UB13} and by Martella at al.~\cite{}, balanced partitioning via linear embedding by Aydin et al.~\cite{}, a streaming technique, called Fennel, suggested by Tsourakakis et al.~\cite{}, and a distributed local search-based algorithm called SocialHash by Kabiljo at al.~\cite{}. We also present results computed by the classical library for graph partitioning, METIS~\cite{bla}.

Table~\ref{table:quality} compares the percentage of cut edges produced by various algorithms. Although the new 
algorithm, \algnameshort, does not always provide the lowest edge cut, it generally produces high quality partitions. The results on the Facebook graphs, shown in Table~\ref{table:qualityFB}, our algorithm
produces best solutions on all but one instance, making it an attractive option for the dataset. We beleive that a careful engineering and tuning of our approach might result in lower edge cuts.

In order to compute results for the largest instances with billions of edges, we implemented \algnameshort in a
vertex-centric programming model and ran experiments in a distributed graph processing system, Giraph\footnote{http://giraph.apache.org}; similar
implementations exist for Spinner~\cite{} and SHP~\cite{}.
In Giraph, a computation is split into supersteps that consists
of processing steps: (i)~a vertex executes a user-defined function based on
local vertex data and on data from adjacent vertices, (ii)~the resulting output
is sent along outgoing edges. Supersteps end with a synchronization barrier, which
guarantees that messages sent in a given superstep are received at the beginning
of the next superstep. The whole computation is executed iteratively for a
certain number of rounds, or until a convergence property is met.

Algorithm~\ref{algo:bp} is implemented in the vertex-centric model with
a simple modification. The first two supersteps compute move gains for all data vertices.
To this end, every query vertex calculates the differences of the cost function
when its neighbor moves from a set to another one. Then, every data vertex
sums up the differences over its query neighbors.
Given the move gains, we exchange the vertices as follows. Instead of sorting the move gains,
we construct, for both sets, an approximate histogram (e.g., as described in~\cite{YE10})
of the gain values.
Since the size of the histograms is small enough, we collect the data on a dedicated
host, and decide how many vertices from each bin should exchange its set. On the
last superstep, this information is propagated over all data vertices and the corresponding
swaps take effect.

As a measure of scalability of our implementation, we report in Figure~\ref{bla} the ...

We observe that 
observe that for the largest instances, \texttt{FB-400B} and \texttt{FB-800B}, the resulting edge cuts of SHP and \algnameshort

Next we will compare the technique against competing tools. (for $d=1$, $\eps=0.03$, $k=2$). We need the following data:

\begin{table}[!h]
\small
    \centering
    \begin{tabular}{lrrrrrr}\toprule
        \multicolumn{1}{c}{Graph} & \multicolumn{1}{c}{\algnameshort} %
        & \multicolumn{1}{c}{SHP} & \multicolumn{1}{c}{LinEm} & \multicolumn{1}{c}{Spinner} & \multicolumn{1}{c}{Fennel} & \multicolumn{1}{c}{METIS} \\
        \midrule
        \texttt{Twitter}
        & $7.3\%$ & $8.33\%$ & $7.43\%$ & $15\%$ & $\boldsymbol{6.8\%}$ & $11.98\%$ \\
        & $\eps=0.02$ & $\eps=0.01$ & $\eps=0.03$ & $\eps=0.05$ & $\eps=0.1$ &$\eps=0.03$ \\
        \midrule
        \texttt{Friendster}
        & $3.73\%$ & $\boldsymbol{3.54\%}$ & $11.9\%$ &  &  &  \\
        & $\eps=0.03$ & $\eps=0.01$ & $\eps=0.03$ &  &  &  \\
        \bottomrule
    \end{tabular}
    \caption{bla.}
    \label{table:quality}
\end{table}

\begin{table}[!h]
\small
    \centering
    \begin{tabular}{lrrr}\toprule
        \multicolumn{1}{c}{Graph} & \multicolumn{1}{c}{\algnameshort} %
        & \multicolumn{1}{c}{SHP} & \multicolumn{1}{c}{machine-hours} \\
        \midrule
        \texttt{FB-2.5B}
        & $\boldsymbol{5.11\%}$ & $8.75\%$ & $1.1$ \\
        \midrule
        \texttt{FB-5.5B}
        & $\boldsymbol{4.99\%}$  & $11.75\%$ & $9$ \\
        \midrule
        \texttt{FB-80B}
        & $\boldsymbol{5.21\%}$  & $12.04\%$ & $13$ \\
        \midrule
        \texttt{FB-400B}
        & $6.88\%$  & $\boldsymbol{5.82\%}$ & $65$ \\
        \midrule
        \texttt{FB-800B}
        & $\boldsymbol{5.52\%}$  & $5.58\%$ & $150$ \\
        \bottomrule
    \end{tabular}
    \caption{bla.}
    \label{table:qualityFB}
\end{table}

-- describe distributed
-- give times

\subsection{Multi-dimensional partitioning}

Here we describe how the alg works for $d>1$. For simplicity we pick $d=2$ and balance on vertices and degrees. Need a plot for:
\begin{itemize}
  \item LiveJournal graph. Quality of "one-dim-GradientDescent vs iterations", 
  "alternating projection vs iterations", "real projection vs iterations".
  Another three plots for "Vertex-imbalance vs iterations".
  Another three plots for "Degree-imbalance vs iterations".
  
  \item com-orkut. (If time permits). Do the same for this graph
\end{itemize}

\subsection{Experiments with projections}
\todo{Dmitry}

First, we need to motivate the projection step. We will do it for $d=1$.
\begin{itemize}
  \item Consider LiveJournal graph. Compute $6$ plots: (i) quality vs iterations, (ii) number of moved vertices vs iterations, (iii) imbalance (max\_vertices/avg\_vertices) vs iterations. First do it for uniform projection ($3$ plots), then for binary-search-based one (another $3$ plots).
  
  \item This one will motivcate the usage of approximate projection. Consider LiveJournal and build a plot "quality vs iterations" for $\eps=0$ (exact projection), $\eps=0.01$ ($1\%$ imbalance), $\eps=0.05$, and $\eps=0.1$.
\end{itemize}


\subsection{Scalability+Distributed computation}
\todo{Sergey}
We'll do it if we have time and space

\input{conclusions}
\newpage
\appendix

\bibliographystyle{alpha}
\bibliography{nips18}

\input{instructions}


\end{document}
